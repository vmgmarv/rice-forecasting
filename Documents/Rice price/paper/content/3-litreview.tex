%----------------------------------------------------------------------------------------
% Literature review
%----------------------------------------------------------------------------------------

\section{Literature review}

There is a growing body of literature on the empirical application of frequency domain
analysis for modeling in the fields of economics and finance. Results thus far have generally been favorable attesting to the usefulness of said techniques for these fields.

One study \cite{konarasinghe2016circular} was conducted to develop a model to forecast Returns of Sri Lankan share market using Fourier transform. The model was named as ``Cirucular model (CM)" which forecasts individual company returns. The study revealed that the CM was successful in forecasting monthly returns of $80\%$ of the companies. They have concluded that CM is a suitable forecasting technique for Sri Lankan share market. However, the authors of the paper have noted that CM is only appropriate if the data shows no trend (detrended).

Another study was conducted by Thomson and Vuuren in 2016 \cite{thomson2016forecasting} to propose a spectral analysis to determine the duration of South African business cycle which is measured by using log changes in nominal gross domestic product (GDP). Three dominant cycles are used to forecast log monthly nominal GDP and forecasts are compared with historical data. The authors found that spectral analysis is more effective in estimating the business cycle length as well as in determining the position of the economy in the business cycle.

A study by Sella et al. in 2016 \cite{sella2010economic} which applies several advanced spectral analysis methods to analyse GDP fluctuations in Italy, Netherlands, and the United Kingdom. They have noted that these analysis tools allowed them to spectrally decompose, as well as reconstruct GDP time series from the data. They further noted that the models are well adapted to the analysis of short and noisy data like the GDP time series in their particular study.

To go further, a study \cite{de2015quantitative} was also conducted to use wavelet transform which is an extensiion of frequency domain models in economics and finanace. This study considered a generalized wavelet that in a particular case of unit amplitudes reduces to the Morlet wavelet. The introduced wavelet appears to be handy for the wavelet processing as it enables even very small details. The study shows some remarkable properties of frequency domain models which make them powerful tools to be used in economics. Thus, it can be seen that there have been attempts to integrate frequency domain analysis in the field of economics.

