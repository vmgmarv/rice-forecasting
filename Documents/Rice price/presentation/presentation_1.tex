%%%%%%%%%%%%%%%%%%%%%%%%%%%%%%%%%%%%%%%%%
% Beamer Presentation
% LaTeX Template
% Version 1.0 (10/11/12)
%
% This template has been downloaded from:
% http://www.LaTeXTemplates.com
%
% License:
% CC BY-NC-SA 3.0 (http://creativecommons.org/licenses/by-nc-sa/3.0/)
%
%%%%%%%%%%%%%%%%%%%%%%%%%%%%%%%%%%%%%%%%%

%----------------------------------------------------------------------------------------
%	PACKAGES AND THEMES
%----------------------------------------------------------------------------------------

\documentclass{beamer}
\usepackage{amsmath}
\mode<presentation> {
\newcounter{savedenum}
\newcommand*{\saveenum}{\setcounter{savedenum}{\theenumi}}
\newcommand*{\resume}{\setcounter{enumi}{\thesavedenum}}
\setbeamertemplate{bibliography item}{\insertbiblabel}


\resetcounteronoverlays{saveenumi}
% The Beamer class comes with a number of default slide themes
% which change the colors and layouts of slides. Below this is a list
% of all the themes, uncomment each in turn to see what they look like.

%\usetheme{default}
%\usetheme{AnnArbor}
%\usetheme{Antibes}
%\usetheme{Bergen}
%\usetheme{Berkeley}
%\usetheme{Berlin}
\usetheme{Boadilla}
%\usetheme{CambridgeUS}
%\usetheme{Copenhagen}
%\usetheme{Darmstadt}
%\usetheme{Dresden}
%\usetheme{Frankfurt}
%\usetheme{Goettingen}
%\usetheme{Hannover}
%\usetheme{Ilmenau}
%\usetheme{JuanLesPins}
%\usetheme{Luebeck}
%\usetheme{Madrid}
%\usetheme{Malmoe}
%\usetheme{Marburg}
%\usetheme{Montpellier}
%\usetheme{PaloAlto}
%\usetheme{Pittsburgh}
%\usetheme{Rochester}
%\usetheme{Singapore}
%\usetheme{Szeged}
%\usetheme{Warsaw}

% As well as themes, the Beamer class has a number of color themes
% for any slide theme. Uncomment each of these in turn to see how it
% changes the colors of your current slide theme.

%\usecolortheme{albatross}
%\usecolortheme{beaver}
%\usecolortheme{beetle}
%\usecolortheme{crane}
%\usecolortheme{dolphin}
%\usecolortheme{dove}
%\usecolortheme{fly}
%\usecolortheme{lily}
%\usecolortheme{orchid}
%\usecolortheme{rose}
%\usecolortheme{seagull}
%\usecolortheme{seahorse}
%\usecolortheme{whale}
%\usecolortheme{wolverine}
\usecolortheme{spruce}
%\setbeamertemplate{footline} % To remove the footer line in all slides uncomment this line
%\setbeamertemplate{footline}[page number] % To replace the footer line in all slides with a simple slide count uncomment this line

%\setbeamertemplate{navigation symbols}{} % To remove the navigation symbols from the bottom of all slides uncomment this line
}

\usepackage{graphicx,caption} % Allows including images
\captionsetup[figure]{labelformat=empty}% redefines the caption setup of the figures environment in the beamer class.

\usepackage{booktabs} % Allows the use of \toprule, \midrule and \bottomrule in tables
\setbeamertemplate{itemize item}{\color{yellow}$\blacksquare$}
\setbeamertemplate{enumerate item}{\color{green}$\blacktriangleright$}

\setbeamerfont{section number projected}{%
	family=\rmfamily,series=\bfseries,size=\normalsize}
\setbeamercolor{section number projected}{bg=black,fg=green}
\setbeamercolor{block body}{bg=green!30,fg=black}
\setlength\abovecaptionskip{-5pt}

\usepackage[natbib=true,style=authoryear,backend=bibtex,useprefix=true]{biblatex}
\addbibresource{literature.bib}

%----------------------------------------------------------------------------------------
%	TITLE PAGE
%----------------------------------------------------------------------------------------

\title[Time series forecasting]{Forecasting rice prices in the Philippines (initial results)} % The short title appears at the bottom of every slide, the full title is only on the title page

\author{Vidal Marvin Gabriel} % Your name
\institute[EFFG] % Your institution as it will appear on the bottom of every slide, may be shorthand to save space
{
Data Analytics Specialist \\
%De La Salle University \\ % Your institution for the title page
\medskip
%\textit{vmgmarv@gmail.com} \\ % Your email address
%\textit{vidal\_gabriel@dlsu.edu.ph}
}
\date{\today} % Date, can be changed to a custom date
\begin{document}

\begin{frame}
\titlepage % Print the title page as the first slide
\end{frame}

\begin{frame}
\frametitle{Outline of the presentation} % Table of contents slide, comment this block out to remove it
\tableofcontents % Throughout your presentation, if you choose to use \section{} and \subsection{} commands, these will automatically be printed on this slide as an overview of your presentation
\end{frame}

%----------------------------------------------------------------------------------------
%	PRESENTATION SLIDES
%----------------------------------------------------------------------------------------

%------------------------------------------------
\section{Introduction} % Sections can be created in order to organize your presentation into discrete blocks, all sections and subsections are automatically printed in the table of contents as an overview of the talk
%------------------------------------------------

%\subsection{Subsection Example} % A subsection can be created just before a set of slides with a common theme to further break down your presentation into chunks

%------------------------------------------------

\begin{frame}
\frametitle{Objective}
\begin{enumerate}
	\item Model rice prices in the Philippines
	\item Forecast rice prices in the Philippines
\end{enumerate}
\end{frame}

%------------------------------------------------
\section{Methodology}
\begin{frame}
	\frametitle{Proposed models}
	\begin{enumerate}
		\item Artificial Neural Networks (Can work with multivariate inputs)
		\item Support Vector Machines (Can work with multivariate inputs)
		\item ARIMA (univariate)
		\item VECM
	\end{enumerate}
\end{frame}

%------------------------------------------------

\begin{frame}
	\frametitle{Data to be used}
	\begin{enumerate}
		\item Total Stock
		\item Household Stock
		\item Commercial Stock
		\item NFA stock
		\item Rice inflation \\
		\textit{Datasource: PSA, 1995 to 2020}
	\end{enumerate}
\end{frame}
%------------------------------------------------
\begin{frame}
\frametitle{Methodology - ANN}
\begin{enumerate}
	\item Inputs:
	\begin{itemize}
		\item For univariate, 3 previous data points = previous 3 months
		\item For multivariate, 3 previous data points and other possible time series (features)
		\\
		Features will be selected based on correlation coefficient.
	\end{itemize}
	\item 1 to 2 hidden layers - will vary based on performance.
	\item Each hidden layer will have 5 to 50 nodes - will vary based on performance.
	\item Activation function - \textit{tanh}; Optimizer - \textit{gradient descent}
	\item 8 iterations/loops will be done to choose the most optimal regressor.
\end{enumerate}
\end{frame}
%------------------------------------------------
\begin{frame}
	\frametitle{Methodology - SVM}
	\begin{enumerate}
	\item Inputs:
	\begin{itemize}
		\item For univariate, previous data point = previous month
		\item For multivariate, previous data point and other possible time series (features)
		\\
		Features will be selected based on correlation coefficient.
	\end{itemize}
		\item Grid search to look for parameters - will look for different combinations of the parameters of SVM
		\item Kernel - radial base function
	\end{enumerate}
\end{frame}
%------------------------------------------------
\begin{frame}
	\frametitle{Methodology - ARIMA}
	\begin{enumerate}
		\item Univariate rice inflation
		\item Grid search to look for parameters - will look for different combinations of the parameters of ARIMA
	\end{enumerate}
\end{frame}
%------------------------------------------------
\begin{frame}
	\frametitle{Methodology - VECM}
	\begin{enumerate}
		\item Lag selection - automated
		\item Check for cointegration
	\end{enumerate}
\end{frame}
%------------------------------------------------

\begin{frame}
	\frametitle{Performance metric}
	\begin{enumerate}
		\item RMSE - 3 data points
		\item MAE - 3 data points
		\item Confusion matrix - proposed metric
	\end{enumerate}
\end{frame}
%------------------------------------------------
\begin{frame}%%     2
	\begin{center}
		{\fontsize{40}{50}\selectfont Results}
	\end{center}
\end{frame}
%------------------------------------------------
\section{Results}
\begin{frame}
	\frametitle{Results}
	\captionsetup{labelformat=empty}
	\begin{table}[h!]
		\begin{center}
			\label{tab:result}
			\resizebox{4cm}{!}{
			\begin{tabular}{c|c|c} % <-- Alignments: 1st column left, 2nd middle and 3rd right, with vertical lines in between
				\textbf{Model} & \textbf{RMSE} & \textbf{MAE}\\
				\hline
				ARIMA & 0.9 & 0.9\\
				VECM & 0.4 & 0.4\\
				SVM\_uni & 1.3 & 1.2\\
				SVM\_mul & 0.3 & 0.3\\
				ANN\_uni & 0.7 & 0.6\\
				ANN\_mul & 0.5 & 0.5\\
			\end{tabular}}
			\caption{\tiny Performance metrics score.}
		\end{center}
	\end{table}
	\captionsetup{labelformat=empty}

\end{frame}
%------------------------------------------------
\begin{frame}
	\frametitle{Confusion matrix}
	\begin{enumerate}
		\item Recall/True positive rate - proportion of actual positives was identified correctly.
		\item Precision - proportion of positive idenfications was actually correct.
		\item F1 score - weighted average of precision and recall.
		\begin{block}{Confusion matrix identifier}Dichotomous data: \textbf{1} - when the model predicted higher inflation than previous inflation and actual inflation is higher than previous inflation; \textbf{0} - when the model predicted lower inflation than previous inflation but in actual inflation is higher than previous inflation.\end{block}
	\end{enumerate}
\begin{table}[h!]
	\captionsetup{labelformat=empty}
	\begin{center}
		\label{tab:result}
		\resizebox{7cm}{!}{
			\begin{tabular}{c|c|c|c} % <-- Alignments: 1st column left, 2nd middle and 3rd right, with vertical lines in between
				\textbf{Model} & \textbf{Recall} & \textbf{Precision} & \textbf{F1}\\
				\hline
				ANN\_uni & 0.8 & 0.6 & 0.7\\
		\end{tabular}}
		\caption{\tiny Confusion matrix trial run.}
	\end{center}
\end{table}
\captionsetup{labelformat=empty}

\end{frame}
%------------------------------------------------
\section{What to do next?}
\begin{frame}
	\frametitle{What to do next?}
	\begin{enumerate}
		\item Include confusion matrix (upon approval)
		\item Documentation
	\end{enumerate}
\end{frame}
\begin{frame}%%     2
\begin{center}
{\fontsize{40}{50}\selectfont Thank You!}
\end{center}
\begin{center}
	\Huge Questions?
\end{center}
\end{frame}


\end{document} 